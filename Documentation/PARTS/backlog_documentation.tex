\chapter{Sprint and Backlog Documentation}\label{cha:sprint_and_backlog_documentation}\thispagestyle{fancy}
This chapter describes how the development workflow and the versioning system is setup. It serves
as an explainer to understand how the project is maintained and a documentation to standardise and
speed up the planing process. The process is described in three parts, preparation, implementation and the
documentation of the sprint after it happens (section~\ref{sec:sprint_planing_and_versioning},~\ref{sec:implementation}
and~\ref{sec:retrospective} respectively). It should be noted that this project follows many common practices of modern
software development, like semantic versioning and dividing the development process into sprints. These practices
are more of a guideline however and will not be followed exactly, as they are simply \emph{overkill} for this project.
They more or less serve as an inspiration and provide loose frameworks to make development more manageable.

\section{Semantic Versioning and Sprint Planing}\label{sec:sprint_planing_and_versioning}
\subsection{Semantic Versioning}
Semantic versioning is a practice used to standardize the way versions get assigned to software by giving each digit meaning. A
version is a three digit number of integers, e.g.: v{\color{red}1}.{\color{blue}2}.{\color{green}0} . Each digit indicates a
different kind of change / version: {\color{red}\emph{major}}, {\color{blue}\emph{minor}}, {\color{green}\emph{patch}}.
In this project, a new version can be called an \emph{increment}. A \emph{major} increment breaks backwards compatibility (e.g:
changing the API). \emph{Minor} increments stay  backwards compatible but add new features to the software. \emph{Patch}
increments fix bugs or add / change smaller details.

\subsection{Sprint Planing}
The term \emph{sprint} is not exactly fitting, it will be more of a marathon. A sprint defines a short time span predefined by the
development team. Due to external factors, the timespan for each sprint won't be predefined. New features will be added once
they are implemented and tested. A sprint in this project should be interpreted as a set of requirements, that the developer tries
to implement next for a given version.

For each sprint, the developers in charge are free to chose which and how many requirements they implement. Try to select
requirements that affect roughly the same area of code (modules, classes, layers, etc...). Before each sprint create a document
(less than one page)\footnote{NOTE: I will create and link a template for this.} that lists the issues / requirements and why the
need to be implemented. If possible, describe the solution that implements these requirements using bullet points or full
sentences. This is not a form submitted for approval and is only for documentation purposes. In case of really small patches or
bug fixes, there is no need to create a separate page. As a rule of thumb: If you would increment the version, create a
documentation.

\section{Implementation}\label{sec:implementation}
\subsection{Coding Guidelines}\label{sec:coding_guidelines}
Flexibility and extendability are the main qualities the code should fulfill (apart from actually \emph{functioning}, that is).
For this reason it is advised to not take to many new features for a new sprint to implement and taking time to fully implement
and test. 

When coding, adhere to the best practices of the used language to handle style. Please refer to an official style documentation.
Also, look at the following chapter~\ref{cha:styleguide} to see how the style is defined or refer to
section~\ref{sec:tools_and_technologies} to get a list of formatters, linters and other tools to ensure a uniform code style.
Please consider the architecture of your code. Softwaredesignpatterns offer standard implementations and solutions for many common
problems. Use them where sensible. Ask the question, if the code is open for extension when another developer needs to come back
to this code and add a feature.

Provide docstrings for the code you write. Write them, so that another developer can use the code only with the informations
provided by the signature and the docstring. Docstrings are used to automatically create and update the documentation for the
code. More on the style of docstrings in chapter~\ref{cha:styleguide}. Unittest your code. Especially the functionality and
any thrown errors should be verified by unittests. Unittests generally adhere to the same guidelines as production code.

\subsection{Branching Strategy}\label{sec:branching_strategy}
The repository will provide two branches: The main branch containing the production code and the development code, where new
features will be integrated and tested. To run a new sprint, create a branch from development and start implementing. Once 
implementation is finished, the code will be tested and integrated into development using a pull request. Pull requests to the
main and development branch need approval from the owner of the repository.

\section{Sprint Retrospective}\label{sec:retrospective}
\subsection{General Retrospective}\label{sec:general_retrospective}
\subsection{More Information}\label{sec:more_information}
After finishing the implementation, create a pullrequest to integrate your changes with the development branch. A pullrequest 
needs to be reviewed and approved by the owner of the repository. If the pullrequest has been approved and integrated into the
development branch, test the application manually to verify that the rest of the application can execute normally.

Create a small retrospective of your development. This is a document\footnote{NOTE: I will create and link a template for this.},
less than one page in size, that shortly explaines the changes / implemented features and lists the implemented requirements /
issues. It is possible to discard some of the requirements that were listed in the planning document. Reasons may be that the
issue was closed unexpectedly, is not relevant or is not possible to implement. Be carefull when discarding requirements and 
consider, if there is a way to implement them. Not enough time \emph{is not} a valid reason to discard an issue and be done
early with the pull request. Take more time and open the pull request later, when the issue is implemented and tesed. If there
have been any, please list some lessons learned at the end of the document.

Please create a document regardless of whether or not a planing document has been created to start the implementation (e.g. a
bugfix)\footnote{If possible, I will provide a smaller document template for bugfixes}. This serves as a documentation and an
overview over what has been done.
